\documentclass[a4paper, 10pt]{article}
\usepackage[utf8]{inputenc}
\usepackage[T1]{fontenc}
\usepackage[scale=0.9]{geometry}
\usepackage[boxed, vlined, french]{algorithm2e}
\usepackage{hyperref}
\usepackage{graphicx}
\usepackage{amsmath}
\usepackage{amssymb}
\usepackage{stmaryrd}
\usepackage{float}
\usepackage{helvet}
\usepackage{libertine}
\usepackage{xcolor}
\usepackage{color}


\begin{document}
\setlength{\parindent}{0cm}
\setlength{\parskip}{1ex plus 0.5ex minus 0.2ex}
\newcommand{\hsp}{\hspace{20pt}}
\newcommand{\HRule}{\rule{\linewidth}{0.5mm}}
\definecolor{orange}{rgb}{0.99,0.69,0.07}

  \title{Modélisation et résolution pour l'optimisation \\ Rapport \\[1ex] \large }
  \author{Par Manon Girard, Paul Peyssard \& Victor Tancrez}
  \date{}
  \maketitle

  \begin{center}
    \includegraphics[scale=0.2]{images/amu2.png}
  \end{center}

  \vfill
  \begin{center}
    Master 2 Intelligence Artificielle \& Apprentissage Automatique \\
    Aix-Marseille Université \\
  \end{center}
  \pagebreak

  \tableofcontents
  \newpage

  \section{Problèmes d'optimisation sous contraintes}

    \subsection{Modélisation du problème}

      Pour chaque instance du problème, nous disposons de diverses informations que nous regroupons par les notations suivantes.
      On note $n$ le nombre de stations et $k$ le nombre de régions de l'instance du problème. De plus, on note $L$ l'ensemble des couples de stations qui souhaitent être en liaisons
      Soit $i \in \{1,2,...,n \}$, on note :
      \begin{itemize}
        \item Pour $ j \in \{1,2,...,n \}, i<j$, $\Delta_{i,j}$ qui correspond à l'écart minimum entre les fréquences des stations $i$ et $j$. Cette valeur est nulle par défaut, et est non nulle lorsqu'elle est renseignée dans les données.
        \item $n_{i}$ le nombre maximum de fréquences différentes utilisées pour la région $i$
        \item $\delta_{i}$ l'écart entre les deux fréquences de la station $i$
        \item $r_{i}$ le numéro de région de la station $i$
      \end{itemize}

      Ainsi, nous pouvons définir l'instance $\mathcal{P} = (X, D, C, f)$ avec :
      \begin{itemize}
        \item $X = \{ fe_i, fr_i | \forall i \in \{1,2,...,n\}$ tel que $\forall i \in \{1,2,...,n\}$:
              \begin{itemize}
                \item $fe_{i}$ correspond à la fréquence pour l'émetteur de la station $i$.
                \item $fr_{i}$ correspond à la fréquence pour le récepteur de la station $i$.
              \end{itemize}
        \item $D = \{d_{fe_i}, d_{fr_i} | \forall i \in \{1,2,...,n\} \}$ où, $\forall i \in \{1,2,...,n\}$
              \begin{itemize}
                \item $d_{fe_i}$ correspond à l'ensemble des valeurs de fréquences possibles pour la fréquence émetrice de la station $i$. Cet ensemble de valeur se retrouve dans le fichier de données.
                \item $d_{fr_i}$ correspond à l'ensemble des valeurs de fréquences possibles pour la fréquence réceptrice de la station $i$. Cet ensemble de valeurs se retrouve dans le fichier de données.
              \end{itemize}
        \item $C = C_1 \cup C_2 \cup C_3 \cup C_4 $ où :
              \begin{itemize}
                \item $C_1$ modélise la contrainte que l'écart entre les deux fréquences d'une même station $i$ doit être $\delta_{i}$.\\
                 On note $C_{1,i} := \{ (d_{fe_i},d_{fr_i})~ tels~que~ | fe_{i} - fr_{i} | = \delta{i} \}$. \\
                 Ainsi, $\displaystyle{C_1 = \bigcup_{i=1}^{n} C_{1,i}}$
                \item $C_2$ modélise l'écart minimum à garantir entre les fréquences de deux stations $i$ et $j$, on note :
                      \begin{itemize}
                        \item $C_{2A,i,j} := \{(d_{fe_i},d_{fe_j},d_{fr_i},d_{fr_j}) ~tels~que~ | d_{fe_i} - d_{fe_j} | \geq \Delta_{i,j} \}$
                        \item $C_{2B,i,j} := \{(d_{fe_i},d_{fe_j},d_{fr_i},d_{fr_j}) ~tels~que~ | d_{fe_i} - d_{fr_j} | \geq \Delta_{i,j} \}$
                        \item $C_{2C,i,j} := \{(d_{fe_i},d_{fe_j},d_{fr_i},d_{fr_j}) ~tels~que~| d_{fr_i} - d_{fe_j} | \geq \Delta_{i,j} \}$
                        \item $C_{2D,i,j} := \{(d_{fe_i},d_{fe_j},d_{fr_i},d_{fr_j}) ~tels~que~ | d_{fr_i} - d_{fr_j} | \geq \Delta_{i,j} \}$
                      \end{itemize}
                      Ainsi, $\displaystyle{C_2 = \bigcup_{i=1}^{n} C_{2A,i,j} \cup C_{2B,i,j} \cup C_{2C,i,j}  \cup C_{2D,i,j}}$
                \item $C_3$ modélise que le nombre de fréquences différentes pour la région $t$ est au maximum $n_t$. \\
                      On note, $C_{3,t} = nValues(\{fr_{i}, fe_{i} | \forall i \in \{1,2,...,n \}, r_{i} = t \}, \leq, n_{t})$. \\
                      Ainsi, $\displaystyle{C_3 = \bigcup_{t=1}^{k} C_{3,t}}$
                \item $C_4$ modélise que si les stations $i$ et $j$ doivent pouvoir communiquer, alors la fréquence émétrice de l'une correspond à la fréquence réceptice de l'autre, et inversement. \\
                On note, $\displaystyle{ C_{4,r} := \{ (d_{fe_i},d_{fe_j},d_{fr_i},d_{fr_j})~tels~que~d_{fr_i} = d_{fe_j}~et~d_{fe_i} = d_{fr_j} \} }$ \\
                Ainsi, $\displaystyle{C_4 = \bigcup_{r=1}^{|L|} C_{4,r}}$

              \end{itemize}

          \item $f$ est la fonction objectif qui va dépendre du cas dans lequel on se trouve. Nous allons donner sa définition dans chaque cas dans les sous-parties suivantes.
      \end{itemize}



      \subsubsection{Cas 1 : Minimisation du nombre de fréquences utilisées}

        Dans le premier cas, nous souhaitons minimiser le nombre de fréquences utilisées. Pour ce faire, nous avons choisi de définir la fonction objectif comme suit :
        $$ \displaystyle{ \min_{n \in \mathbb{N}}  nValue(\{ e_{i}, r_{i} | \forall i \in \{1,2,...,n \}, =, m) }$$

        Intuitivement, la contrainte globale $nValue$ permet de fixer le nombre de valeurs de fréquences différentes, émétrice et réceptrice mélangées, à $m$. Ainsi, on souhaite minimiser la valeur de $m$, ce qui correspond à notre problème.

      \subsubsection{Cas 2 : Utilisation des fréquences les plus basses}

        Dans ce deuxième cas, l'objectif est d'utiliser les valeurs de fréquences les plus basses possibles. Pour ce cas, nous avons eu deux points de vue différent. Ainsi, nous avons défini deux fonctions objectif différentes. Les voici :
        \begin{itemize}
          \item On parlera ici du $CAS 2$. On défini la fonction objectif suivante :
                $$ \displaystyle{\min_{n \in \mathbb{N}} \sum_{i=1}^{n} fe_i + fr_i } $$

                Intuitivement, cette fonction cherche à minimiser la somme totale des fréquences, en encourageant l'utilisation des fréquences les plus basses pour chaque station.

           \item On parlera dans ce cas du $CAS 2BIS$. On défini la fonction objectif par :
                 $$ \displaystyle{\min \{ \max \{ \max_{i} fe_i , \max_{i} fr_i \} \}} $$
                 Intuitivement, on prend la plus grande valeur des fréquences émétrice, de même pour les fréquences réceptrice, on prend la plus grande des deux et on tente de la rendre la plus petite possible. Ainsi, on prend la valeur de fréquence la plus grande et on tente de la minimiser, ce qui va forcer toutes les autres à être minimisées.
        \end{itemize}

      \subsubsection{Cas 3 : Minimiser la largeur de la bande de fréquences utilisées}

        Dans ce troisième et dernier cas, on souhaite minimiser la largeur de la bande de fréquences utilisées, c'est-à-dire, l'écart entre la plus basse et la plus haute fréquence. Intuitivement cela se traduit par la fonction objectif suivante :
        $$ \displaystyle{ \min_{} | \max_{} \{\max_{j} fe_j, \max_{j} fr_j\}  - \min_{} \{\min_{j} fe_j, \min_{j} fr_j\}  |} $$

    \subsection{Résultats d'expériences}

      \subsection{Valeur}

      Est-ce que ça vaut le coup de donné les valeurs des fonctions objectif ?? (si oui je te ferai le tableau) Mais genre on dit quoi dessus mdr ?

      En vrai ça peut rajouter quelque chose en plus mais pas obligé. Peut être pour montrer dans chaque cas un compromis temps d'execution/valeur de fonction objectif et faire une petite phrase dessus

      \subsubsection{Temps d'execution}

      Nous avons réuni dans les tableaux suivant les temps d'exécution pour les 4 cas différentes. Le tableau de gauche correspond aux temps d'execution lorsqu'on utilise le solveur choco, tandis que le tableau de droite correspond au solveur ace. Il est important de préciser que nous avons limité le temps d'exécuton à 10 minutes.

      \begin{minipage}[t]{0.5\linewidth}
        \footnotesize
        \begin{tabular}{ |c|c|c|c|c| }
          \hline
          \textbf{Nom du fichier} & \textbf{CAS 1} & \textcolor[gray]{0.6}{\textbf{CAS 2}}   & \textbf{CAS 2 BIS} & \textbf{CAS 3} \\
          \hline
          \scriptsize{150\_13\_15\_5\_0.800000\_26} & \textcolor{red}{9,114s} & \textcolor[gray]{0.6}{ 10m0,003s} & 1,875s& \textcolor{green}{1,775s}\\
          \hline
          \scriptsize{150\_13\_15\_5\_0.800000\_28} & \textcolor{red}{8,746s} & \textcolor[gray]{0.6}{ 10m0,003s} & \textcolor{green}{1,768s} & 2,177s \\
          \hline
          \scriptsize{150\_13\_15\_5\_0.800000\_29} &\textcolor{red}{ 23,106s} & \textcolor[gray]{0.6}{ 10m0,003s} & \textcolor{green}{2,432s} & 9,770s \\
          \hline
          \scriptsize{150\_13\_15\_5\_0.800000\_2} & \textcolor{red}{9,103s} & \textcolor[gray]{0.6}{ 10m0,003s} & 3,863s & \textcolor{green}{4,869s} \\
          \hline
          \scriptsize{150\_13\_15\_5\_0.800000\_8} & 14,385s & \textcolor[gray]{0.6}{ 10m0,006s} & \textcolor{green}{8,735s} & \textcolor{red}{14,841s}\\
          \hline
          \scriptsize{250\_25\_15\_5\_0.820000\_20} & \textcolor{red}{20,851s} & \textcolor[gray]{0.6}{ 10m0,003s} & \textcolor{green}{12,317s} & 15,637s \\
          \hline
          \scriptsize{250\_25\_15\_5\_0.820000\_22} &\textcolor{red}{ 21,368s} & \textcolor[gray]{0.6}{ 10m0,014s} & \textcolor{green}{10,138s} & 15,675s \\
          \hline
          \scriptsize{250\_25\_15\_5\_0.820000\_5} & \textcolor{red}{24,417s} & \textcolor[gray]{0.6}{ 10m0,004s} & \textcolor{green}{13,985s} & 18,861s \\
          \hline
          \scriptsize{250\_25\_15\_5\_0.820000\_7} & \textcolor{red}{1m25,918s} & \textcolor[gray]{0.6}{ 10m0,009s} & \textcolor{green}{17,085s} & 23,151s \\
          \hline
          \scriptsize{250\_25\_15\_5\_0.820000\_9} & \textcolor{red}{30,516s} & \textcolor[gray]{0.6}{ 10m0,004s} & \textcolor{green}{12,427s} & 16,081s\\
          \hline
          \scriptsize{500\_30\_20\_5\_0.870000\_24} & \textcolor{red}{1m5,376s} & \textcolor[gray]{0.6}{ 10m0,008s} & \textcolor{green}{22,674s} & 32,296s \\
          \hline
          \scriptsize{500\_30\_20\_5\_0.870000\_29} & \textcolor{red}{1m48,111s} & \textcolor[gray]{0.6}{ 10m0,009s} & \textcolor{green}{23,076s} & 1m4,631s \\
          \hline
          \scriptsize{500\_30\_20\_5\_0.870000\_45} & \textcolor{red}{1m37,911s} & \textcolor[gray]{0.6}{ 10m0,008s} & \textcolor{green}{18,963s} & 44,055s \\
          \hline
          \scriptsize{500\_30\_20\_5\_0.870000\_48} & \textcolor[gray]{0.6}{10m0,016s} & \textcolor[gray]{0.6}{ 10m0,017s} & \textcolor{green}{21,797s} & \textcolor{red}{50,926s} \\
          \hline
          \scriptsize{50\_7\_10\_5\_0.800000\_0} &\textcolor{red}{ 15,259s} & \textcolor[gray]{0.6}{ 10m0,003s} & \textcolor{green}{7,826s} & 10,754s \\
          \hline
          \scriptsize{50\_7\_10\_5\_0.800000\_1} & 14,113s & \textcolor[gray]{0.6}{ 10m0,067s} & \textcolor{green}{8,736s} & \textcolor{red}{21,748s} \\
          \hline
          \scriptsize{50\_7\_10\_5\_0.800000\_6} & \textcolor{red}{12,067s} & \textcolor[gray]{0.6}{ 10m0,071s} & \textcolor{green}{6,799s} & 7,601s \\
          \hline
          \scriptsize{50\_7\_10\_5\_0.800000\_7} & \textcolor{red}{28,409s} & \textcolor[gray]{0.6}{ 10m0,006s} & \textcolor{green}{7,375s} & 8,325s \\
          \hline
          \scriptsize{50\_7\_10\_5\_0.800000\_8} & \textcolor{red}{13,511s} & \textcolor[gray]{0.6}{ 10m0,003s }& 8,190s & \textcolor{green}{7,660s} \\
          \hline
        \end{tabular}
        ~\\
        \centering
        \textbf{\large Tableau des résultats avec le solveur Choco} % C'est le titre de la minipage
      \end{minipage}
      \hfill
     \begin{minipage}[t]{0.6\linewidth}
       \footnotesize
         \begin{tabular}{ |c|c|c|c|c| }
           \hline
           \textbf{Nom du fichier} & \textbf{CAS 1} &\textcolor[gray]{0.6}{ \textbf{CAS 2}} & \textbf{CAS 2 BIS} & \textbf{CAS 3} \\
           \hline
           \scriptsize{150\_13\_15\_5\_0.800000\_26} & \textcolor{red}{9,148s} & \textcolor[gray]{0.6}{10m0,003s} & \textcolor{green}{1,934s} & 2,080s\\
           \hline
           \scriptsize{150\_13\_15\_5\_0.800000\_28} & \textbf{\textcolor{red}{7,527s}} & \textcolor[gray]{0.6}{10m0,003 }& \textcolor{green}{2,045s} & 2,819s \\
           \hline
           \scriptsize{150\_13\_15\_5\_0.800000\_29} & \textcolor{red}{38,114s} & \textcolor[gray]{0.6}{10m0,003s} & \textcolor{green}{3,033s} & \textbf{8,701s} \\
           \hline
           \scriptsize{150\_13\_15\_5\_0.800000\_2} & \textcolor{red}{11,311s} & \textcolor[gray]{0.6}{10m0,003s} & \textcolor{green}{6,704s} & 8,018s \\
           \hline
           \scriptsize{150\_13\_15\_5\_0.800000\_8} & \textcolor{red}{19,314s} & \textcolor[gray]{0.6}{10m0,003s} & \textcolor{green}{9,039s} & 16,969s \\
           \hline
           \scriptsize{250\_25\_15\_5\_0.820000\_20} & \textcolor{red}{22,871s} & \textcolor[gray]{0.6}{10m0,003s} & \textcolor{green}{12,827s} & 15,720s \\
           \hline
           \scriptsize{250\_25\_15\_5\_0.820000\_22} & \textbf{\textcolor{red}{20,711s}} & \textcolor[gray]{0.6}{10m0,040s} & \textcolor{green}{13,023s} & 17,823s \\
           \hline
           \scriptsize{250\_25\_15\_5\_0.820000\_5} & \textbf{\textcolor{red}{21,584s}} & \textcolor[gray]{0.6}{10m0,003 }& \textbf{\textcolor{green}{13,005s}} & \textbf{15,649s} \\
           \hline
           \scriptsize{250\_25\_15\_5\_0.820000\_7} & \textbf{\textcolor{red}{1m19,278s}}& \textcolor[gray]{0.6}{10m0,004s} &\textbf{ \textcolor{green}{11,586s}} & \textbf{21,511s} \\
           \hline
           \scriptsize{250\_25\_15\_5\_0.820000\_9} & \textbf{\textcolor{red}{29,120s}} & \textcolor[gray]{0.6}{10m0,017s} & \textcolor{green}{14,780s} & 20,984s \\
           \hline
           \scriptsize{500\_30\_20\_5\_0.870000\_24} & \textcolor{red}{1m17,392s} & \textcolor[gray]{0.6}{10m0,069s} & \textcolor{green}{23,233s} & 36,927s \\
           \hline
           \scriptsize{500\_30\_20\_5\_0.870000\_29} & \textcolor{red}{1m37,635s} & \textcolor[gray]{0.6}{10m0,008s} & \textcolor{green}{18,379s} & 49,842s \\
           \hline
           \scriptsize{500\_30\_20\_5\_0.870000\_45} & \textcolor{red}{1m42,731s} & \textcolor[gray]{0.6}{10m0,018s} & 1m2,116s & \textcolor{green}{51,214s} \\
           \hline
           \scriptsize{500\_30\_20\_5\_0.870000\_48} & \textcolor[gray]{0.6}{10m0,011s} & \textcolor[gray]{0.6}{10m0,009s} &  \textcolor{green}{22,408s} & \textbf{\textcolor{red}{43,302s}} \\
           \hline
           \scriptsize{50\_7\_10\_5\_0.800000\_0} &\textbf{ \textcolor{red}{12,524s}} & \textcolor[gray]{0.6}{10m0,003s} & \textcolor{green}{8,766s} & \textbf{10,333s} \\
           \hline
           \scriptsize{50\_7\_10\_5\_0.800000\_1} & \textbf{13,005s} & \textcolor[gray]{0.6}{10m0,003s} & \textcolor{green}{8,972s} & \textbf{\textcolor{red}{19,113s}} \\
           \hline
           \scriptsize{50\_7\_10\_5\_0.800000\_6} & \textcolor{red}{37,418s} & \textcolor[gray]{0.6}{10m0,073s} & \textbf{\textcolor{green}{6,258s}} & \textbf{7,170s} \\
           \hline
           \scriptsize{50\_7\_10\_5\_0.800000\_7} & \textbf{\textcolor{red}{23,807s}} & \textcolor[gray]{0.6}{10m0,007s} & \textcolor{green}{8,416s} & \textbf{10,006s} \\
           \hline
           \scriptsize{50\_7\_10\_5\_0.800000\_8} & \textbf{\textcolor{red}{12,348s}} & \textcolor[gray]{0.6}{10m0,068s} & \textbf{7,770s} & \textcolor{green}{7,660s} \\
           \hline
         \end{tabular}
       ~\\\\
       \centering
       \textbf{\large Tableau des résultats avec le solveur Ace} % C'est le titre de la minipage
     \end{minipage}

     Nous avons mis en gris les temps correspondant à des non-aboutissements, c'est-à-dire que le processus a été arrêté en cours d'exécution, au bout de 10 minutes.

     \begin{itemize}
       \item Concernant le \textit{CAS 2}, nous observons une absence totale de résultats aboutis. Cette situation contraste fortement avec le \textit{CAS 2 BIS}, qui a systématiquement abouti à des résultats. Cette différence pourrait suggérer que la fonction objectif définie dans le \textit{CAS 2} est soit mal adaptée, soit trop complexe pour être traitée efficacement par le solveur utilisé. Cette distinction entre les deux cas souligne l'importance d'une formulation adéquate de la fonction objectif dans la résolution de problèmes de programmation sous contrainte.
       \item Généralement, on remarque que le solveur Choco est plus efficace en terme de temps que le solveur Ace. Les valeurs dans le tableau de droite sont en gras lorsque le temps d'exécution est plus faible en utilisant le solveur Ace qu'en utilisant le solveur choco.
       \item On peut voir que dans la plus part des instances, le \textit{CAS 2 BIS} à le plus faible temps d'exécution.
     \end{itemize}



  \section{Problèmes de satisfaction de contraintes valués}

  \subsection{Modélisation du problème}

    Pour chaque instance du problème, nous disposons de diverses informations que nous regroupons par les notations suivantes.
    On note $n$ le nombre de stations et $k$ le nombre de régions de l'instance du problème. De plus, on note $L$ l'ensemble des couples de stations qui souhaitent être en liaison.
    Soit $i \in \{1,2,...,n \}$, on note :
    \begin{itemize}
      \item Pour $ j \in \{1,2,...,n \}, i<j$, $\Delta_{i,j}$ qui correspond à l'écart minimum entre les fréquences des stations $i$ et $j$. Cette valeur est nulle par défaut, et est non nulle si elle est renseignée dans les données.
      \item $n_{i}$ le nombre maximum de fréquences différentes utilisées pour la région $i$
      \item $\delta_{i}$ l'écart entre les deux fréquences de la sation $i$
      \item $r_{i}$ le numéro de région de la station $i$
    \end{itemize}

    Ainsi, nous pouvons définir l'instance $\mathcal{P} = (X, D, W, SV)$ avec :
    \begin{itemize}
      \item $X = \{ fe_i, fr_i | \forall i \in \{1,2,...,n\}$ tel que $\forall i \in \{1,2,...,n\}$:
            \begin{itemize}
              \item $fe_{i}$ correspond à la fréquence pour l'émetteur de la station $i$.
              \item $fr_{i}$ correspond à la fréquence pour le récepteur de la station $i$.
            \end{itemize}
      \item $D = \{d_{fe_i}, d_{fr_i} | \forall i \in \{1,2,...,n\} \}$ où, $\forall i \in \{1,2,...,n\}$
            \begin{itemize}
              \item $d_{fe_i}$ correspond à l'ensemble des valeurs de fréquences possible pour la fréquence émetrice de la station $i$. Cet ensemble de valeur se retrouve dans le fichier de données.
              \item $d_{fr_i}$ correspond à l'ensemble des valeurs de fréquences possible pour la fréquence réceptrice de la station $i$. Cet ensemble de valeurs se retrouve dans le fichier de données.
            \end{itemize}
      \item $W$ :
            \begin{itemize}
              \item L'écart entre les deux fréquences d'une même station doit être $\delta_{i}$ pour toute sation $i$ :
               $| fe_{i} - fr_{i} | = \delta{i}$ $\rightarrow$ \color{green} 0 \color{red} + $\infty$ \color{black}

               \item L'écart minimum à garantir entre les fréquences des stations $i$ et $j$. Pour cette contrainte, on a différencié le cas où les stations $i$ et $j$ sont en liaisons et le cas où elles ne le sont pas :
               \begin{itemize}
                 \item Si $i$ et $j$ sont en liaisons (on est face à des contraintes dures) :
                       \begin{itemize}
                         \item $| fe_i - fe_j | \geq \Delta_{i,j} \rightarrow \textcolor{green}{0}, \textcolor{red}{+\infty}$
                         \item $| fe_i - fr_j | \geq \Delta_{i,j} \rightarrow \textcolor{green}{0}, \textcolor{red}{+\infty}$
                         \item $| fr_i - fe_j | \geq \Delta_{i,j} \rightarrow \textcolor{green}{0}, \textcolor{red}{+\infty}$
                         \item $| fr_i - fr_j | \geq \Delta_{i,j} \rightarrow \textcolor{green}{0}, \textcolor{red}{+\infty}$
                       \end{itemize}
                  \item Si les stations $i$ et $j$ ne sont pas forcément en liaison, on passe sur des contraintes douces car on estime que les interferences ont moins d'impacte puisque les stations ne rentrent pas en communication. Le poids d'une contrainte violée est $\Delta_{i,j}$ car on souhaite que plus l'écart demandé est grand, plus la contrante doit être satisfaite.
                        \begin{itemize}
                          \item $| fe_i - fe_j | \geq \Delta_{i,j} \rightarrow \textcolor{green}{0}, \textcolor{red}{\Delta_{i,j}}$
                          \item $| fe_i - fr_j | \geq \Delta_{i,j} \rightarrow \textcolor{green}{0}, \textcolor{red}{\Delta_{i,j}}$
                          \item $| fr_i - fe_j | \geq \Delta_{i,j} \rightarrow \textcolor{green}{0}, \textcolor{red}{\Delta_{i,j}}$
                          \item $| fr_i - fr_j | \geq \Delta_{i,j} \rightarrow \textcolor{green}{0}, \textcolor{red}{\Delta_{i,j}}$
                        \end{itemize}
               \end{itemize}

               \item Si les stations $i$ et $j$ doivent pouvoir communiquer, on impose que la fréquence réceptrice de l'une soit la fréquence émetrice de l'autre, et inversement. On est donc face à une contrainte dure :
               $fr_i = fe_j$ et $fe_i = fr_j$ $\rightarrow$ \textcolor{green}{0}, \textcolor{red}{+$\infty$}
            \end{itemize}
        \item $SV = (\mathbb{N} \cup \{ + \infty \}, \leq, +, 0, +\infty)$
    \end{itemize}



  \section{Consignes}

    Quelques expériences avec 2 ou 3 solveurs --> faire un retour d'expérience,
    Avec Toulbar faire avec différents réglages pour voir des comportement différents
    Analyser les résultats --> étude de cas dans la globalité : voir les différences entre les solveur etc Voir que le critère a une incidence sur le temps de calcul etc, faire un conclusion sur qui est le meilleur tout ça

    Sur la partie expérimentale : le temps est important ! Limité le temps à 10 minutes

    Rapport + achives avec sources mettre un seul fichier XML par types de trucs (pour pas que ce soir trop lourd)

\end{document}
