\documentclass[a4paper, 10pt]{article}
\usepackage[utf8]{inputenc}
\usepackage[T1]{fontenc}
\usepackage[scale=0.8]{geometry}
\usepackage[boxed, vlined, french]{algorithm2e}
\usepackage{hyperref}
\usepackage{graphicx}
\usepackage{amsmath}
\usepackage{amssymb}
\usepackage{stmaryrd}
\usepackage{float}
\usepackage{helvet}
\usepackage{libertine}
\usepackage{color}


\begin{document}
\setlength{\parindent}{0cm}
\setlength{\parskip}{1ex plus 0.5ex minus 0.2ex}
\newcommand{\hsp}{\hspace{20pt}}
\newcommand{\HRule}{\rule{\linewidth}{0.5mm}}
\definecolor{orange}{rgb}{0.99,0.69,0.07}

  \title{Modélisation et résolution pour l'optimisation \\ Rapport \\[1ex] \large }
  \author{Par Manon Girard \& Paul Peyssard \& Victor Tancrez}
  \date{}
  \maketitle

  \begin{center}
    \includegraphics[scale=0.2]{images/amu2.png}
  \end{center}

  \vfill
  \begin{center}
    Master 2 Intelligence Artificielle \& Apprentissage Automatique \\
    Aix-Marseille Université \\
  \end{center}
  \pagebreak

  \tableofcontents
  \newpage

  \section{Problèmes d'optimisation sous contraintes}

    \textbf{Les données} $\forall i \in \{1,2,...,n \}$ et $\forall j \in \{1,2,...,n \}$
    \begin{itemize}
      \item $n$ := le nombre de stations
      \item $k$ := le nombre de région
      \item $\Delta_{i,j}$ := l'écart minimum entre les fréquence des stations $i$ et $j$ (possiblement nul)
      \item $n_{i}$ := le nombre maximum de fréquences différentes utilisées pour la région $i$
      \item $\delta_{i}$ := l'écart entre les deux fréquences de la sation $i$
      \item $r_{i}$ := le numéro de région de la station $i$
    \end{itemize}

    \textbf{Les variables : } $\forall i \in \{1,2,...,n \}$
    \begin{itemize}
      \item $fe_{i}$ := la fréquence pour l'émetteur de la station $i$
      \item $fr_{i}$ := la fréquence pour le recepteur de la station $i$
    \end{itemize}

    \textbf{Les contraintes} $\forall i,j \in \{1,2,...,n \}$, et $\forall t \in \{1,2,...,k\}$
    \begin{itemize}
      \item L'écart entre les deux fréquences d'une même station doit être $\delta_{i}$ :
       $| fe_{i} - fr_{i} | = \delta{i}$
      \item L'écart minimum à garantir entre les fréquences des stations $i$ et $j$ :
      \begin{itemize}
        \item $| fe_i - fe_j | \geq \Delta_{i,j}$
        \item $| fe_i - fr_j | \geq \Delta_{i,j}$
        \item $| fr_i - fe_j | \geq \Delta_{i,j}$
        \item $| fr_i - fr_j | \geq \Delta_{i,j}$
      \end{itemize}
      \item Le nombre de fréquence différentes pour la région $t$ est au maximum $n_t$ : $nValues(\{fr_{i}, fe_{i} | \forall i \in \{1,2,...,n \}, r_{i} = t \}, \leq, n_{t})$
      \item Si les stations $i$ et $j$ doivent pouvoir communiquer:
      $fr_i = fe_j$ et $fe_i = fr_j$
    \end{itemize}

    \subsection{Cas 1 : Minimisation du nombre de fréquence utilisées}

      Dans ce cas, la fonction objective est :
      $$ \displaystyle{ \min_{n \in \mathbb{N}}  nValue(\{ e_{i}, r_{i} | \forall i \in \{1,2,...,n \}, =, n) }$$

    \subsection{Cas 2 : Utilisation des fréquences les plus basses}

      Dans ce cas,
      $$ \displaystyle{\min_{n \in \mathbb{N}} \sum_{i=1}^{n} fe_i + fr_i } $$

      Option2
      Dans ce cas,
      $$ \displaystyle{\min \{ \min \{ \max_{i} fe_i , \max_{i} fr_i \} \}} $$

    \subsection{Cas 3 : Minimiser la largeur de la bande de fréquence utilisées}

      Dans ce cas,
      $$ \displaystyle{ \min_{} | \max_{} \{\max_{j} fe_j, \max_{j} fr_j\}  - \min_{} \{\min_{j} fe_j, \min_{j} fr_j\}  |} $$

  \section{Problèmes de satisfaction de contraintes valués}

  \textbf{Les contraintes} $\forall i,j \in \{1,2,...,n \}$, et $\forall t \in \{1,2,...,k\}$
  \begin{itemize}
    \item L'écart entre les deux fréquences d'une même station doit être $\delta_{i}$ :
     $| fe_{i} - fr_{i} | = \delta{i}$ -> \color{green} 0 \color{red} + $\infty$ \color{black}
    \item L'écart minimum à garantir entre les fréquences des stations $i$ et $j$ (Dures pour toutes les stations qu'on veut en liaison et grosse molles pour le reste ) :
    \begin{itemize}
      \item $| fe_i - fe_j | \geq \Delta_{i,j}$
      \item $| fe_i - fr_j | \geq \Delta_{i,j}$
      \item $| fr_i - fe_j | \geq \Delta_{i,j}$
      \item $| fr_i - fr_j | \geq \Delta_{i,j}$
    \end{itemize}
    \item Le nombre de fréquence différentes pour la région $t$ est au maximum $n_t$ : $nValues(\{fr_{i}, fe_{i} | \forall i \in \{1,2,...,n \}, r_{i} = t \}, \leq, n_{t})$ 0 pour inférieur à $n_t$ et après ensuite 1 à chaque nouvelle station (donner un maximum pour pas faire exploser le nombre de fréquence)
    \item Si les stations $i$ et $j$ doivent pouvoir communiquer:
    $fr_i = fe_j$ et $fe_i = fr_j$  \color{green} 0 \color{red} + $\infty$ \color{black}
  \end{itemize}

  Contraintes dures 2 pour toutes les stations devan être en liaisons.

  \subsection{Cas 2 }

    \textbf{Les contraintes} $\forall i,j \in \{1,2,...,n \}$, et $\forall t \in \{1,2,...,k\}$
    \begin{itemize}
      \item L'écart entre les deux fréquences d'une même station doit être $\delta_{i}$ :
       $| fe_{i} - fr_{i} | = \delta{i}$ -> \color{green} 0 \color{red} + $\infty$ \color{black}
      \item L'écart minimum à garantir entre les fréquences des stations $i$ et $j$ :
      \begin{itemize}
        \item $| fe_i - fe_j | \geq \Delta_{i,j}$
        \item $| fe_i - fr_j | \geq \Delta_{i,j}$
        \item $| fr_i - fe_j | \geq \Delta_{i,j}$
        \item $| fr_i - fr_j | \geq \Delta_{i,j}$
      \end{itemize}
      \item Le nombre de fréquence différentes pour la région $t$ est au maximum $n_t$ : $nValues(\{fr_{i}, fe_{i} | \forall i \in \{1,2,...,n \}, r_{i} = t \}, \leq, n_{t})$
      \item Si les stations $i$ et $j$ doivent pouvoir communiquer:
      $fr_i = fe_j$ et $fe_i = fr_j$
      \item On veut maximiser le nombre de liaison :  $\displaystyle{ \max_{n} nValues(\{ (i,j) | i<j, fe_i = fr_j fe_j = fr_i \}, =,n)}$ -->
  \end{itemize}

  \section{Consignes}

  Quelques expériences avec 2 ou 3 solveurs --> faire un retour d'expérience,
  Avec Toulbar faire avec différents réglages pour voir des comportement différents
  Analyser les résultats --> étude de cas dans la globalité : voir les différences entre les solveur etc Voir que le critère a une incidence sur le temps de calcul etc, faire un conclusion sur qui est le meilleur tout ça

  Sur la partie expérimentale : le temps est important ! Limité le temps à 10 minutes

  Rapport + achives avec sources mettre un seul fichier XML par types de trucs (pour pas que ce soir trop lourd)

\end{document}
